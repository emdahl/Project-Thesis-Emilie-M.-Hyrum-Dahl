\chapter{Conclusion and Further Work}
\label{chap:conclusion}
The research question asked at the beginning of this thesis was... This chapter will give a conclusion answering the research question, pulling strings from the result and discussion. In this project thesis, one objective was to do the necessary preparations for a master thesis that will be written and delivered in the spring of 2019. With this in mind, a tentative plan for further work, is expressed at the end of this chapter. 


\section{Conclusion}
The research question asked at the beginning of this thesis was: \textit{Is it possible to identify plastics using a spectrometer in the visual spectrum?} Throughout experimental work and analysis, the answer has revealed itself. A correlation and clustering within the provided data was found. However, these were based on the lightness and color of the objects, not the IOP from material composition. As neither the Principal Components Analysis nor interpretation of the signatures directly, revealed a distinct signature for plastics, it was not possible to distinguish individual types of plastic nor plastic in general with the use of spectroscopy.
\\\\
Based on the presented results and discussion it has become clear that identifying plastics based on hyperspectral imaging in the visible spectrum is not possible at this point. 

\section{Further Work}
This project is a part of a larger scheme of achieving continuous monitoring of the microplastic concentration of selected areas of the ocean. Many small pieces need to be sorted out an put in place in order for this to happen. The following chapter is intended to summarize the additional work and obstacles required to obtain the next stage laying the grounds for the grander goal. 
\\\\
Apart from teasing about combining silhouette and hyperspectral imaging in Chapter \ref{chap:today}, these two technologies have been kept strictly apart. As the wide range in chemistry, and subsequent properties, of the microplastics leave such a varied fragmentation pattern, the silhouette camera alone could find it hard to detect plastic. However, in combination with a spectrometer, elaborated throughout this thesis, it might give grounds for detection. The combination of measurements from the two sensors could possibly combine to form an identification tool. However, this needs to be further investigated and tested properly, as the spectrometer might not give expected results due to the conclusion of this paper. 
\\\\
Detection of plastic underwater requires a platform carrying the sensors across the water columns in the best way possible. In chapter \ref{chap:today}, alternative UUVs were discussed, ending with the AUV as the possibly most efficient platform for this purpose. One of the next steps in this process is therefore to implement a functional detection sensor on an AUV, testing them both underwater.

\section{Obstacles}
In order to accomplish the objective of identifying marine microplastics \textit{in situ}, one needs to overcome several obstacles along the way. The work and investigation for the project have uncovered several possible issues with recognizing particles. These are either due to the plastic interacting with the marine environment, such as marine snow or growth, the decomposition of plastic, additives due to the wide range of use of plastic causing large internal variations, or a combination of the above. Either way, the identified issues must be addressed when going further.

\subsection{Decomposition}
As color will bleach over time, it might be possible that by the time the plastic have reached the state categorized as microplastics, the surface color of the plastic has been bleached. However this is not a constant transition as this process would greatly depend on the buoyancy of the particle. An issue would be that the light in water is weaker than in the air, meaning a less reactive environment underwater. This means that the color might be conserved for a longer amount of time underwater. This poses as a problem due to the significant effect of the inherent color of the plastic seen in the laboratory testing. Due to the difference in decomposition and subsequent bleaching, it is hard to determine the exact concentration. A possible solution would be to only identify the bleached plastic and use it as a measurement for the total concentration by means of regression. However, due to light exposure, composition, growth etc., different colors and different plastic types will bleach at different rates. This will make it difficult to determine the age and abundance of the microplastic and subsequently use it as an indicator.

\subsection{Variations in Plastics}
Because of local environmental differences, there is no guarantee that plastic of the same type will break down in the same fashion. Even though the plastic is of a certain type, the manufacturer may add additional substances in order to change the properties of the plastic. Thus, all plastics of type PET will not necessarily have the same properties, and might possibly not react in the same fashion.
\\\\
Differences in the plastic composition will make it difficult to recognize the plastic. This could either be due to the composition of the plastic itself, the additives added or simply the deterioration the piece has been subjected to over its lifetime. First, there are a plethora of different types and compositions of plastic. One common denominator is therefore crucial in order to recognize. Secondly, two samples of the same plastic type does not necessarily contain the same substances. Depending on the intended use for the plastic there are a numerous additives added to it. This can done to give it different properties and functionality, i.e., harder, softer or porous, or to give the plastic a certain design by adding colors. As the method of the report is heavily dependent on color signatures, this might pose an issue. 
\\\\
Another issue is that the proposed set-up with the SilCam recognition, is dependant on the shape of the plastic particles. This shape is rarely the same for the different particles, as the shape is also likely to be affected by the chemical composition of the plastic and the environmental wear and tear.
\\\\
The hyperspectral imaging technique requires the surface to be visible in order to acquire the correct signature. If the surface is covered by growth or other substances, making the surface inaccessible, it will not be possible to retrieve anything from the spectroscopy of the plastic. In such cases, a possible solution could be to combine a spectrometer with a SilCam lens, as previously presented. However, the SilCam might have problems with the transparency of the material. If the flash penetrates the substance in such a fashion that the silhouette is ruined, the method will be impossible to use. Also, the SilCam is based on a given shape required to be somewhat constant in-between samples. This is hard to come by in the case of plastic. As mentioned, the different types and various additives give different properties and will deteriorate in a different fashion. The difference in deterioration and properties will result in non-consistent shapes.
\\\\
However, there might still be certain aspects of the shapes only related to plastics. Unnatural shapes might pose a solution. Man-made objects tend to have different composition and subsequently different shapes than natural objects. The objects might have straight or flat edges, spikes or other recognizable aspects of the shape making it possible to exclude from natural occurring particles and subsequently classify as plastic.

\subsection{Environmental Effects}
The difference in light environment might cause disturbance on the signatures. Due to the fact that rays from the sun penetrating different environments at different degrees, the measurements must be done using a dominating portable light source in order to have a controllable and somewhat constant light signature at sampling. A possible solutions would be to use a similar technique as the SilCam suggests. The sample volume is constrained in controlled environment while being measured. Thus, the light environment would be somewhat controllable. 
\\\\
However, the lamp will create noise as the light of the lamp itself will have its own signature. If not absolutely certain about the signature of this light, correct recognition will not occur. This would be possible to solve using calibration. However, the light bulbs have a limited lifetime. Despite the differences between individual bulbs, light from the same source, i.e., containing the same gas, should have the same signature. In this case, the type of light only needs to be calibrate once. The variations from this will therefore be negligible, as local variations in light will be more significant. 
\\\\
Lastly, as mentioned, growth on plastic particles might affect the shape of the particle. One could argue that even though the plastic deteriorates in a plethora of shapes, one could still recognize the unnatural shape of a man-made material. The decomposition might not result in individual shapes distinct for the material, but there could still be patterns in the fragments unique to plastic. However, the growth on the plastic might make these features less prominent. The result could be an unrecognizable shape, not possible to classify as a plastic. 