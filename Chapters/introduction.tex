\chapter{Introduction}
\label{chap:introduction}
This chapter will introduce the background motivating the work done in this thesis. The tremendous amount of plastic entering the ocean is affecting the marine environment to a large degree. The first step in the journey towards a plastic free ocean is finding ways to detect microplastic. This chapter will highlighting the main findings the author has obtained in this area. Lastly, an outline of the rest of the thesis will be presented.

\section{Motivation}
%and Background??

%The background of the study will discuss your problem statement, rationale, and research questions. It links introduction to your research topic and ensures a logical flow of ideas.  Thus, it helps readers understand your reasons for conducting the study.
%HER MÅ JEG OGSÅ HA MED AT SOM NEVNT I ABSTRACT??, MÅ VI GJENNOM FLERE STEG. NESTE STEG ER Å DESIGNE EN SENSORBÆRENDE PLATFORM SOM KAN BRUKE DETEKSJONSMETODENE. 

%http://ec.europa.eu/environment/marine/good-environmental-status/descriptor-10/pdf/GESAMP_microplastics%20full%20study.pdf

Today, the applications of plastics are many, making the material popular worldwide. %335 million metric tons of plastic was produced in 2016. %(https://www.statista.com/statistics/282732/global-production-of-plastics-since-1950/)
% https://www.darrinqualman.com/global-plastics-production/
Approximately 400 million metric tons of plastic is produced yearly, and the production is projected to nearly double within the next 10-15 years \cite{worldenvironmentday_2018}. In addition to being a strong, light and inexpensive material, the different types of plastic cover an almost full-scaled spread of needs. High electrical isolation properties are useful in one area, while durable and strong material can be ideal in other areas. What can be read of this? Plastics have incredibly useful and versatile properties. However, this worldwide spread of plastics holds a significant side effect. 
\\\\
The different types of plastic have polymer structures that make the material almost non-biodegradable. In addition, the structures of plastics can hold additives. This is initially something that can be utilized, as additives can be incorporated in order to give the plastic a desired property. Typical additives are fire retardants, stabilizers, antibiotics, pigments etc. \cite{qualman_2017} Problems arise when the additives that blend into the pieces of plastics, turn out to be toxins and the plastic debris serves as a vector, introducing toxic elements into the ecosystem. As this vector is not biodegradable, it can travel between ecosystems as an immortal catalyst.
\\\\
The largest source of microplastic is the discharge of larger plastic pieces gradually fragmented into smaller plastic debris. The fragmentation occurs due to wear and tear from the environment, but the biggest cause of decomposition is UV radiation from the sun. The rays provide oxidative decomposition of polymers, which in turn will give weak and brittle plastic debris. At this state, mechanical forces can easily fragment the weak plastic piece. Of such forces wind, waves and human activity are good examples. The state of a large plastic piece floating around the many seas can thereby quickly result in many smaller pieces and in turn be fragmented to microplastic. In addition, wind and currents will bring the microplastic away from its initial point of outlet, leaving the microplastic widely spread.
\\\\
The largest issue with microplastic is the ingestion of marine biota. Once microplastic is ingested, it is in most cases retained in the digestive system or absorbed by the intestines \cite{browne_galloway_thompson_2007}. After a while, the plastic debris are stored in organs and tissue. Due to the fact that plastics are not biodegradable, the chemicals will hardly be excreted by the organism ingesting the piece, but rather accumulate. This is called bioaccumulation, \cite{bart}. 
\\\\
%https://www.sciencedirect.com/science/article/pii/S0160412017322298 
Bioaccumulation creates the foundation of biomagnification, causing the poisoning of entire food chains. Organisms and animals placed at a higher trophic level are often in need of a larger amount of food than what the species on a lower level can serve. This consumption need means that the concentrated bioaccumulation is even larger at higher trophic levels \cite{maddison_carbery_wayne_o'connor_palanisami_thavamani_2018}. In fact, the concentration of an environmental toxicity increases with each level of the food chain - in turn affecting human beings as well. This is called biomagnification. 
\\\\
Microplastic can be confused with phyto-plankton. This is especially inconvenient as the phyto-plankton creates the basis of all biomass production in the ocean. Phyto-plankton is eaten by the zoo-plankton, which again is food to larger organisms. One of the most common types of microplastic, polystyrene, can often be mistaken for phyto-plankton \cite{sussarellu_suquet_2016}. This is tragic, as an intake of polystyrene could come with a series of negative consequences. %(http://www.pnas.org/content/113/9/2430).
\\\\
An example of a typical bioaccumulated chemical, traveling through trophic levels, is Polychlorinated biphenyls (PCBs), a group of manmade chemicals. In 1988 is was proved that PCB in the bird's tissue originates from plastic particles \cite{derraik_2002}. Sadly PCBs are harmful chemicals, even in very low amounts. The ingestion of PCB, for instance through a zoo-plankton containing PCB-loaded microplastics, can cause reproductive disorders, change the hormone levels and increase the risk of several diseases. In some cases, an intake can lead to death. %[(Ryan et al., 1988, Lee et al., 2001)].
\\\\
Furthermore, some types of bacteria are attracted to free floating marine debris. These are better known as “hitch hikers”, and can threaten sensitive coastal environments, as the bacteria are far from their natural habitats, turning out even more toxic in unknown territory, \cite{gregory_2009}. 
%(Environmental implications of plastic debris in marine settings-entanglement, ingestion, smothering, hangers-on, hitch-hiking and alien invasions Murray R. Gregory)
\\\\
Different types of plastic have different densities. Some types have higher density than water and thereby float, while other types are denser than water. This contributes to the fact that plastic can be found throughout the entire ocean column, and is present in many ecosystems \cite{derraik_2002}. As a result, plastic debris have impacted more than 690 marine. Small particles of plastic have been observed in the digestive tract of organisms from different trophic levels. 
\\\\
While microplastics does damages as a toxic carrying vector, larger pieces of plastics can act as a physical threat in other ways. Polyethylene bags, operating in the ocean currents, have a large resemblance to the predators targeted by turtles. Plastic debris can thus prevent their survival.
The ingestion of larger plastic debris can, among other things, reduce food intake, cause internal damage, strangling or even death after blockage of the intestinal tract, \cite{derraik_2002}.
\\\\
Marine biological diversity is already exposed to climate change, over-fishing and other man-made disruptions. As if this were not enough, plastic pollution, with more than eight billion kilograms entering the ocean annually \cite{yurtoglu_lykketoft_2016}, and thereby over five trillion pieces of plastic currently littering the ocean \cite{the_ocean_cleanup}, the marine environment is exposed to huge amounts of damage. \cite{derraik_2002}
%* = The pollution of the marine environment by plastic debris: a review Jose G.B. Derraik *

\section{Research Question}
Now that the large impact of plastic is clear, methods on how to reduce this impact are in order. A good way to work with materials, identify them or learn about their properties, is to study how light interacts with them - spectroscopy. By definition spectroscopy examines how light behaves in the target and recognizes materials based on their different spectral signatures. This can be identified from the spectrum, describing the amount of light in different wavelengths, showing how much light is reflected, emitted and transmitted from the target. In other words, the spectrum simply shows how much of a certain color the light contains. 
\\\\
Spectral signatures can be thought of as fingerprints. While fingerprints often is used to identify a person, spectral signatures can be used to identify materials. One of the purposes of this study, is to identify different types of plastic using their spectral signatures. %A thought: As different types of plastic are composed by different polymer structure accepting additives in various degree, one might be able to identify the plastic types most resistant to additives
\\\\
%Development of hyperspectral imaging as a bio-optical taxonomic tool for pigmented marine organisms - geir
Former studies conducted on marine organisms, show that reflectance signatures, obtained from an underwater hyperspectral imager, are related to the absorption signature of that specific organism. \cite{pettersen_johnsen_bruheim_andreassen_2013} %(Volent et al. 2007, 2009)
\\\\
This leads to the research questions driving this project, centered around whether microplastics and their spectral signatures in fact can be separated from other material and microorganisms using a hyperspectral camera. Do different types of plastics leave different spectral signatures? Will a signature change as the sun and sea tears on the plastic piece? Is it possible that the spectral signatures differ in different environments?
\\\\
The latter questions are important to address when detecting and inspecting microplastic in the ocean. This thesis will dive deeper into particularly one question:\\ \textbf{Is it possible to identify plastics using a spectrometer in the visual spectrum?}

\section{Main Contributions}
Through the work done in this thesis, the author have concluded that within the visual light spectra, the hyperspectral camera cannot capture the material composition of the targeted objects. This means that under these conditions, different plastics cannot be identified - neither as plastics in general, nor as specific types, separated from each other.

\\\\
\section{Thesis Outline}
This section describes the outline of the thesis. Figure \ref{fig:outline} is a visual description trying to catch the relation within and between the chapters presented below. 
\subsubsection*{CHAPTER 1} Chapter 1 is an introduction containing the motivation behind the research question, and a short answer to this question, highlighting the main contributions of the work. 
\subsubsection*{CHAPTER 2} Chapter 2 presents an overview of today. What has been done so far? What type of sensors and platform carrying these sensors have been proven sufficient?
\subsubsection*{CHAPTER 3} Chapter 3 takes care of the fundamentals of optics, describing in detail the properties of light and its areas of usage.
\subsubsection*{CHAPTER 4} Chapter 4 introduce and elaborate imaging, the imaging process and different aspects of it.
\subsubsection*{CHAPTER 5} Chapter 5 describes the data creating the base of this thesis, and how this data can be computed and analyzed.
\subsubsection*{CHAPTER 6} Chapter 6 takes care of the method for the experimental work done in this thesis.
\subsubsection*{CHAPTER 7} Chapter 7 present, compare and discuss the results obtained from the experiment and corresponding analysis.
\subsubsection*{CHAPTER 8} Chapter 8 is a short conclusion pulling strings from the result and discussion. In addition, ideas on further work are presented here.
\\\\\\
\begin{figure}[H]
  \includegraphics[width=\linewidth]{Images/thesisoutline.png}
  \caption[Thesis Outline]{Thesis outline, roughly sketched}
  \label{fig:outline}
\end{figure}


