\chapter{Discussion}
\label{chap:discussion}
 In this chapter several aspects of the project thesis will be discussed.
 
 %Klassifisering basert på PCA-resultatene
 
\section{Signatures Based on the PCA}
As previously mentioned, it was not possible to acquire distinct signatures for the individual types of plastic. Neither was is possible to find a general signature for the plastic which could be used as an end-member or for Spectral Angle Mapping. Despite seeing a resemblance of a signature in the second principal component it is important to keep in mind the components does not represent actual signatures. They rather depict the wavelengths which have a larger influence on the vertical distinguishing of the samples.
\\\\
Based on the absence of clear clusters it was deemed not possible to acquire what is thought of as unique signatures for the plastic. All three analyses do show vertical spread, but the large number of overlapping samples made it difficult to argue for the existence of clear signatures at this stage. In the analysis with the presence of the organic pigments, there are some indication of separation, but not enough argue for a general signature for the plastic. However, the clustering seen in the sample does pose an opportunity of distinguishing plastic from the natural occurring pigments. 
 
\section{Signatures Directly Interpreted}
In order to obtain a color-free spectrum, one could subtract the color spectrum from the resulting spectrum retrieved from the experiments. However, most pieces of plastic do not have one specific color, making it difficult to determine the spectrum to subtract. 
\\\\
In addition. When experimenting with different types of plastics having more or less the same colors, the results turned out next to identical. At first, a thought was that this was due to the color being too dominant in relation to the rest of the properties. However, conducting the same experiment with non-colored pieces of plastic, gave the same end-result; no difference in spectral signature even if the types of plastics differ. - could this be due to reflectance disturbance from "shiny" see through material, or is it in fact impossible to distinguish the different types from each other using hyperspectral imaging in visible light?
\\\\
Discussing the size-difference. Also, why is the intensity this different?
 
\section{Sources of Error}
Due to the ambient environment and sensors related to Hyper Spectral Imaging noise will be next to impossible to exclude. There is always some dark noise due to the sensors of the system. Also, since the measured values are based of a ratio rather than absolute values small changes could have large outcomes due to the fact that the values are divided by extremely small values. A change in the divisor may be small, but the outcome large due to the relative change. However, the tests were limited to the visible spectrum, whereas the biggest variation were in the infrared or near-infrared spectrum. Even though there were variations in the visible spectrum the results were satisfactory with respect to change due to noise. 
\\\\
Furthermore, the ambient light in the room is hard to completely avoid. Even though all light was turned off and blinds were used some light still got in. The light will affect the results, and could possibly affect the signature. However, due to the low amount of lighting that were present during the tests, it is assumed that it did not have a considerable effect on the results. The ambient light is therefore assumed to be negligible. 
\\\\
Even though the plastic was order specially in order to have pure samples, there are still sources that are unknown. As a result, the results could be affected by the differences in the plastic composition made by the manufacturer. However, the results showed a clear tendency where the actual color of the plastic was dominant as opposed to the chemical composition.
\\\\
The see-through plastic will be affected by the background. Compensating for the paper was not performed, and the results were most likely is affected by the background. The effect is not completely clear, but the trend of see through plastic have a similar peak to paper is could display the effect of the background. On the other hand, the background would most likely affect the results of the clear plastic either way. Therefore, the results were deemed satisfactory with the white background.
\\\\
The scanning was easily affected by the movement of the scanner. This was portrayed in the change of the reflectance standard. It is therefore a concern that the same effect will have caused the signatures to be inconsistent. However, the reflectance was regularly updated and there was several measurements per plastic type in order to account for large individual variations. It is still a concern that single measurements that were affected could skew the results.
\\\\


\section{Nothing but thoughts for now..}
VISUAL LIGHT:
Experiments analyzing the wave spectra of different types of plastics, have already been conducted. However, the results has been directed towards viewing the wave length interval describing near infrared light (NIR). Although the use of infrared light is an effective means of unobtrusive observation on land, it is far less effective in the ocean because long wavelength light is rapidly attenuated by seawater. One can therefore discuss whether it is possible to detect microplastic underwater with NIR. Is it possible to get close enough? etc..
\\\\
%https://oceanoptics.com/plastic-recycling-nir-spectroscopy/
Another important observation in these experiments, is the condition of the microplastic used. The material is pure and white, making the results independent of color. 
\\\\
%Water absorption 
%https://commons.wikimedia.org/wiki/File:Absorption_spectrum_of_liquid_water.png
%This logarithmic (log-log) graph shows water’s absorption behavior at different colors wavelength. As seen in the graph, water absorption is minimized between 400 -600 nm


%Light Transmission in the Ocean: http://www.waterencyclopedia.com/La-Mi/Light-Transmission-in-the-Ocean.html
%https://manoa.hawaii.edu/exploringourfluidearth/physical/ocean-depths/light-ocean

\\\\
\noindent
AN IDEA: 
What if we could inspect the level of additive acceptance in the different types of plastic. By finding the most resistant types, it might be possible to modify these into doing the same job as the types accepting additives. This way, we could reduce the production of the largest vectors/toxin-carriers. 