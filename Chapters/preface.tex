%\thesistitlepage % make the ordinary titlepage
\hypersetup{pageanchor=true}
%\chapter*{Summary}

\chapter*{Preface}
This thesis is submitted in partial fulfillment of the requirements for the degree of master of science (MSc) at the Norwegian University of Science and Technology (NTNU). The main work is conducted at Department of Marine Technology, NTNU, while part of the work has been conducted at the Department of Biology, NTNU in cooperation with Aksel Alstad Mogstad and Geir Johnsen, Department of Biology. 
\\\\
I have cooperated with Andreas {\O}. R. Stien in large part. Next to all testing and work done hands on, have been done together with Andreas. Therefore we share more or less the same \textit{Method} and \textit{Results and Discussion}. In addition, Andreas and I will work together for the masters thesis starting next year. The section regarding further work should therefore also be equally written. 
\\\\
The work supporting this thesis is three-folded. In order to achieve relevant and precise knowledge, a literature study was conducted. This study involved research on both specific methods and technology used, and what has been done in general. The research regarding these topics were done mainly through researching a bunch of scientific papers. However, even with the most recent publications, there are still research done - not yet published. The second part of this thesis was therefore to travel around Trondheim meeting with the experts. During exciting meetings with Emlyn Davies, Andy M. Booth, Geir Johnsen, Aksel Alstad Mogstad, Bert van Bavel, Asgeir J. S{\o}rensen and Albert Van Oyen - from now on referred to as \textit{the experts} - new knowledge was acquired. After this, the problem description finally took form. Through planning with the experts, the experiment description were formed. These experiments represent the third and last part of the three-folded work. 
\\\\
This semester, I have had two very relevant module courses, \textit{TTK19 Structures and Contexts in Complex Systems} -  a course on how to handle Quantitative Big Data, and \textit{TTK20 Hyperspectral remote sensing} - describing the hyperspectral imaging process. Throughout these courses, I have gained much relevant knowledge, some reflected in this thesis.


\begin{comment}
\newline
\newline

\newline
\newline
\newline
\newline
\newline
\newline
\begin{center}
    Trondheim, November 14, 2018
    \end{center}
\begin{figure}[H]
\centering
%\includegraphics[scale=0.5]{figures/sign}
\end{figure}
\begin{center}
Emilie Dahl
\end{center}
\end{comment}