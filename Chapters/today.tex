\chapter{Overview of Sensors and Sensor Carrying Platforms}
\label{chap:today}

\begin{comment}
Marine biological diversity is already exposed to climate change, overfishing and other man-made disruptions. As if this were not enough, plastic pollution also causes huge damage to the marine environment. (*)
\\
335 million metric tons of plastic was produced in 2016. %(https://www.statista.com/statistics/282732/global-production-of-plastics-since-1950/)
It is projected that the production will nearly double within the next 10-15 years
%https://wedocs.unep.org/bitstream/handle/20.500.11822/25398/WED%20Messaging%20Two-Page%2027April.pdf?sequence=12&isAllowed=y
\\\\

Kanehiro et al. (1995) states that plastic accounted for 80-85 percent of the seabed waste in Tokyo Bay in the 90's. (*)
This is a striking finding, considering that most plastic residues are floating to some degree. Different types of plastic have different densities. Some types have higher density than water and float, while other types are denser than water. This contributes to the fact that plastic can be found throughout the sea column, and is present in many ecosystems. For instance, plastic have been found in the digestive system of organisms of all sizes, from small marine invertebrates to whales.
\\\\
Polyethylene bags, operating in the ocean currents, have a large resemblance to the predators targeted by turtles. (*) Plastic debris can thus prevent their survival (Bugoni et al., 2001, Duguy et al., 1998). The ingestion of plastic debris can, among other things, reduce food intake, cause internal damage, strangling (hvordan bøyes dette) or even death after blockage of the intestinal tract. (Zitko and Hanlon, 1991).
\\\\
Plastic is also a potential carrier of chemicals and can absorb bacteria already present in the ocean, including polychlorinated biphenyls (PCBs). (*) PCBs are harmful chemicals, even in very low amounts. The ingestion of PCB can cause reproductive disorders, change the hormone levels and increase the risk of several diseases. In some cases, an intake may lead to death [(Ryan et al., 1988, Lee et al., 2001)]. Ryan et al. (1988) proved that PCB in the bird's tissue originates from plastic particles. Plastic pellets can thus be transporter for PCB in marine food chains (Mato et al., 2001).
\\\\
Furthermore, different types of bacteria are attracted to free floating marine debris. These are better known as “hitch hikers”, and can threaten sensitive coastal environments, as the bacteria are far from their natural habitats. 
%(Environmental implications of plastic debris in marine settings-entanglement, ingestion, smothering, hangers-on, hitch-hiking and alien invasions Murray R. Gregory)
\\\\
Some phyto-plankton eating species are particularly exposed, as microplastic easily can be confused with phyto-plankton. One of the most common types of microplastic, polystyrene (PS), has shown to affect the ability to reproduce. %(http://www.pnas.org/content/113/9/2430).
%* = The pollution of the marine environment by plastic debris: a review Jose G.B. Derraik *
\end{comment}
%\section{Today}


\noindent
Solving today’s plastic problem is not an easy task - especially with plastics and microplastics being present not only in the water surface, but in the entire water column and seabed. Imagine multiplying the entire ocean surface with a few hundred meters dept. This leaves an almost impossible starting point. In addition, there are continuous currents and circulations, allowing a large spread. On top of it all, the deep sea is difficult to reach, and if reaching it, poor sight is often a fact. So, what have been done so far in order to solve this plastic problem?
\\\\
The problem with plastic is becoming an increasingly known problem. Since plastic contamination and many of its consequences are visible to the naked eye, few people can deny that the pollution of plastic is a fact. Nevertheless, it is not enough to be aware of the problem – the world must cooperate and act. Today, a number of initiatives already exist, not only carrying a physical effect, but hopefully motivating and catalyzing a new time. REV Ocean is an initiative by Kjell Inge R{\o}kke, who, together with Nina Jensen, have established REV Ocean to \textit{contribute to saving the world’s oceans}. The idea is research ships, with three of the purposes behind it being acquiring knowledge, creating awareness and collecting plastic waste \todo{https://revocean.org/}. The Ocean Cleanup is another initiative, solely determent of collecting plastics from the five largest garbage patches in the world, using currents to catch and concentrate the plastic \todo{https://www.theoceancleanup.com/}. Contests around the world, with a purpose of finding sustainable solutions are increasingly present. One of them are the Plastic Ocean Project, challenging people to find \textit{innovate novel ways to “mine” for plastics}. \todo{http://www.plasticoceanproject.org/fishing-4-plastic.html}.
\\\\
Many of these initiatives are focusing on the areas with the largest amount of plastics - the garbage patches, with the largest being the Great Pacific Garbage Patch. \todo{legge inn garbage patch?} \todo{https://www.theoceancleanup.com/} These areas are a natural place to begin, as the concentration is at its most. Nevertheless, the entire ocean needs to be mapped. %(Evidence that the Great Pacific Garbage Patch is rapidly accumulating plastic 2018, L. Lebreton1,2, B. Slat1, F. Ferrari1 – mer herfra). 

\section{Hyperspectral Imaging}
One obvious requirement when mapping and cleaning the ocean of plastic, is the availability of an instrument able to detect plastic in the ocean separating it from the rest of the ocean particles. This is where the field of spectroscopy enters the court. The development of image detectors, especially the two-dimensional silicon charge coupled device (CCD), has revolutionized image spectroscopy. CDD provides information on the distribution of photon intensity along the spectrograph's entrance slit. The distribution of entrance slit into different wavelengths and intensities, has made it possible to reconstruct detailed images at high spatial (defined as 1 cm) and spectral (defines as 1 nm) resolution. This makes a hyperspectral imager particularly suitable, as it consists of a spectrometer equipped with this charge coupled device (CCD). 
% (Development of hyperspectral imaging as a bio-optical taxonomic tool for pigmented marine organisms - geir)
\\\\
%Use of Underwater Hyperspectral Imager (UHI) in Marine Archaeology 2014, Øyvind Ødegård1,3, Geir Johnsen2 and Asgeir J. Sørensen1
Hyperspectral imagery is defined as images that contain a spectrum of reflected light with a spectral resolution of 1-5 nm per image pixel. The object of interest, from now on denoted OOI, will absorb, scatter and reflect light of different portions of the visible spectrum. This will in sum give them their unique optical fingerprints, which can be used for classification with high degree of confidence.
\\\\
%(BASIC HYPER SPECTRAL IMAGING F. SIGERNES)
%G. Vane, ed., Imaging Spectroscopy II, Proc. SPIE 834, 1988.
%W.L. Wolfe, Introduction to Imaging Spectrometers, Vol. TT25 of Tutorial Text Series, SPIE Press, Bellingham, Wash., pp. 1-147, 1997.
The last decade, most of the work and discovery in hyperspectral imaging, have been within space and air-born sensors. Here, the instrument has proven to be a particularly powerful remote sensing control tool. It is not until recently that the technology has been tested underwater by Ecotone, which is the first mover with patents for underwater application. \todo{kilde, slides 102}
\\\\
\todo{ref geir johnsen et al. 2015} As the sea is not stagnant and objects in water can behave differently than in air, new challenges will occur. Nevertheless, research shows that the technology is promising, also underwater, and can be a very useful tool in many areas. The technology allows a withdrawal of information on dimensions, material, surface shape, color, pigments, orientations and position. This information will in turn lay the grounds for the creation of high resolution spatial models of underwater scenes, retrieved from optical imagery. \todo{kilde:marreg1 86}. Together, this information is the spectral signature of an object. Depending on the object's material, color and pigments, the signature will vary. This way, it is possible to identify various objects. As an example, Ecotone have manage to separate living and dead tissue in underwater sea weed, due to difference in reflectance. \todo{kilde:marreg1 103}
\\\\
%In addition, work on detection and classification of plastic in the sea, has been conducted, using hyperspectral imaging underwater. \todo{mer fra bert sitt arbeid}
%Bert: http://journals.sagepub.com/doi/pdf/10.1255/jnirs.1212 (mer herfra). 
Nevertheless, as mentioned, challenges do occur. The large number of image frames necessary of hyperspectral data requires iterative calculations on high dimensionality data, which has large computational costs \todo{http://docsdrive.com/pdfs/ansinet/itj/2008/1030-1036.pdf}. In addition, going underwater the attenuation and scattering of light can be dominant, leaving useful rays at wastage. Shadow patterns underwater can contribute to shifts in the imagery, making the extracted image less representative when describing the surroundings. \todo{marreg1, 86}
\\\\

%https://oceanoptics.com/plastic-recycling-nir-spectroscopy/  !!!

% https://www.researchgate.net/profile/Hamed_Masoumi/publication/285330830_Identification_and_classification_of_plastic_resins_using_near_infrared_reflectance_spectroscopy/links/572af79808aef7c7e2c5026d/Identification-and-classification-of-plastic-resins-using-near-infrared-reflectance-spectroscopy.pdf?origin=publication_detail 
\noindent
 Regarding plastic detection, experiments analyzing the wave spectra of different types of plastic, have already been conducted. Among others, Hamed Masoumi et. al. \todo{legge inn kilden over}, have demonstrated that identification and separation of plastic types in fact can be obtained, using measurements of the near infrared light (NIR) reflectance spectroscopy. Of the research done in this thesis, all results resulting in plastic identification, have been directed towards viewing the wave length interval describing NIR. Although the use of infrared light is an effective means when doing observation on land, light is rapidly attenuated by seawater which makes this approach far less effective when taking the problem below surface.
\\\\
Therefore, experiments using hyperspectral imaging in the spectra of visual light is on the agenda for this thesis. 

\section{SilCam}
The Silhouette camera (SilCam) developed by Emlyn Davies at SINTEF, is also interesting in this matter. This technology is different, as it utilizes a light source behind the object to be identified to clearly see the outline of the target. Simplified, the hyperspectral imagery extracts the contents of what is depicted, while the SilCam captures the exterior. The hyperspectral and the silhouette camera seems to compliment each other. Could a combination of the two possibly give an even better way of detecting microorganisms?

%JEG ER VELDIG USIKKER PÅ OM DENNE BØR KOMME HER, SOM EN TEASER TIL NOE SOM EGENTLIG KOMMER I FURTHER WORK OG DERMED I NESTE RUNDE...


\section{Sensor Carrying Platform}
%ulike sensorer for deteksjon og kartlegging (vise sammenhengen til dette og en bredere oversikt - dette blir isåfall bare her, men tas ikke med videre - i metode kan vi heller si at vi avgrenser mot cam. + aktuelle sensorer

In this project, the purpose is to lay the grounds for a continuous autonomous monitoring of the oceanic microplastics concentration. In order to perform underwater detection and mapping, a sensor carrying platform is needed. %(Imaging sensors attached to an underwater platform are not necessarily used for object identifying alone, but is also commonly used in orientation and localization of the platform itself.) 
In Appendix \ref{app:platform}, a figure of some sensor platform's temporal and spatial resolution and coverage, is displayed.
\\\\
As mentioned earlier, plastics are circling the entire ocean, including the ocean surface, water columns and seabed. Platforms like Unmanned Aerial Vehicle (UAV), often used as a sensor carrier in on-land missions, can therefore be out-ruled. When further deciding on a suitable platform to carry the sensors, Unmanned Underwater Vehicles (UUV), are typically relevant, including platforms such as Autonomous Underwater Vehicle (AUV), Remotely Operated Vehicle (ROV) and Gliders.
\\\\
ROVs are robots that are remotely maneuvered from a control room from either a ship or a platform. These provide detailed mapping and sampling in the target area, with high resolution data on the target. Th umbilical attached to the ROV at one end and the ship at the other end, gives unlimited electrical power and high bandwidth communication. However, the need of this umbilical limits the spatial coverage (usually lower than 1000 $m^2$, Appendix \ref{app:platform}). 
\\\\
An AUV however, is independent of an umbilical, and can operate without ships or associated platforms present. Similar to the ROV, the AUV has a high spatial resolution data, providing detailed seafloor and water column mapping. In addition, the covered area per time is more than ten times the range of an ROV. From the figure in Appendix \ref{app:platform}, some properties of the Glider can be extracted. It seems that the Glider have an even larger spatial range. Nevertheless, comparing the AUV to a Glider, the wide payload capacity of the AUV and the high spatial resolution data, makes the AUV a more suitable choice for the desired purpose described above. 
