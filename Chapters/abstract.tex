\chapter*{Abstract}

The ocean plays a great part in life on earth, not only as a source to oxygen and food for living beings, but also as a vital influence to the climate and weather. The ocean, with all that comes with it, is simply a necessity for life on earth. Nevertheless, this same ocean is not taken care of. Millions of pieces of plastic, especially microplastic, pollutes and eradicate sea life every day, indirectly affecting life on land. 
\\\\
The road to a clean sea can be divided into steps. First step is to prevent the continuous supply of trash and plastic pollution. The next step should involve picking up what is already in the ocean. The latter action is further divided into sub-steps. As the ocean is huge, attacking all parts of the sea is next to impossible. Being able to map and monitor the ocean columns and determine critical areas is thus a good start. Therefore, this thesis will concern methods for detection and analyzing microplastics. 
\\\\
A hyperspectral imager has been used to extract information from five types of microplastics. Based on the data acquired, spectral signatures from each plastic type was plotted. The interpretation of the spectral signatures was that the color of the object was too dominant to separate based on anything else than color. Principal component analyses was then ran, confirming the previous observation. However, a closer interpretation of the results recovered the true governing factor - the \textit{lightness} of the object was even more dominant, determining the variation in large part. 
\\\\
Through the work done in this thesis, the author have concluded that within the visual light spectra, the hyperspectral camera cannot capture the material composition of the targeted objects. This means that under the given conditions, different plastics cannot be identified - neither as plastics in general, nor as specific types, separated from each other.

%An abstract is a brief summary of a research article, thesis, review, conference proceeding, or any in-depth analysis of a particular subject and is often used to help the reader quickly ascertain the paper's purpose.

\hypersetup{pageanchor=false}